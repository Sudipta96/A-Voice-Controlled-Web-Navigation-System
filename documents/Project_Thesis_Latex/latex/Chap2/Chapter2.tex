
\chapter{Literature Review}
	\label{chapter:2}
	\section{Overview}
	Before initiating the development of the proposed voice controlled chrome extension, a detail study was conducted based on related existing tools, research works, extensions. This study gives a proper insight which helped to build this project. The study demonstrates the features of the existing works and review them about their limitations. It highlighted how this tools approached to implement voice recognition, natural language understanding and how to interact with the browser which forms the foundation for developing a enhanced proper voice controlled chrome extension with extensive features.
	
	The study also shows the importance of the response time, performance, user experience, supported languages of those works, By studying these works and papers, the opportunities of new features/solution have been identified.
	
	\section{Related Works}
	
	\subsection{Browser-based Voice Extensions}
	\textbf{Handsfree for Web \cite{handsfree}} provides a lots of set features including web search(Google, Wikipedia), click links, scrolling, form filling, media control, text reading and some basic browser control features. The strength of this system lie in extensive features and multilingual support(4 languages). But it has no integration with Gmail, Google Docs, or map related features.
	
	\textbf{Readme – TTS \cite{readme}} mainly focuses on the accessibilty of the browser supported by over 50 languages. It helps user to listen to text from webpage, emails and PDF. But this extension does not support browser control or media control. The limitation of this extension is less features, mainly focuses on text-speech features rather than browser interaction.
	
	\textbf{Hey Buddy – Chrome Voice Assistant\cite{heybuddy}} provides features for basic browsing interaction and media control including search, scrolling and link navigation. It is lightweight and easy to use but it does not support Bangla and does not provide some advanced features like email management and Google Doc interaction or custom commands.
	
	\textbf{LipSurf - Voice Control for the Web\cite{lipsurf}} allows user to interact with browser using voice commands. It supports tab control, form filling and media control. It provides a good set of features but does not integrate Bangla language to communicate.
    
	\textbf{Voice Typing\cite{voicetyping}} enables users to write text based on voice input. It receives voice input from users and translates it into text by using text-to-speech API. It does not come with other features like browser control or media control. Its scope is limited to type text only.
	
	\textbf{Speech Recognition Anywhere\cite{speechrec}} offers flexible features with multiple language support. It allows features to copy, paste and basic form filling. It supports browser control features but does not include advanced productivity tools. 
	
	Other notable extensions include: 
	\begin{itemize}
		\item	Text-to-Speech that Brings Productivity\cite{texttospeechprod} – TTS focused with limited control features.
		\item	Voice In Voice Typing\cite{voicein} – Dictation tool can be used for fill in forms and email writing specially.
		\item	Dictation for Gmail\cite{dictationgmail} – Specialized for email accessibility. 
		\item	Talkie: Text-to-Speech\cite{talkie} – Multilingual reading tool for selected text.
		\item	NaviVoice: Voice Input Productivity Assistant\cite{navivoice} – Limited browsing and form filling features are offered by this extension.
		\item	Voice Actions for Chrome (beta)\cite{voiceactions} – Simple browser commands(only able to search through query)
		\item	Say Play\cite{sayplay} – Media playback control. This extension is only used to control media content.
		\item	Neo\cite{neo} – Lightweight browsing and limited productivity features. Google searching by opening new site is feature here.
		\item	Speech to Text (Voice Recognition)\cite{speechtext} – Dictation focused.
		\item	Capti Voice\cite{captivoice} – Accessibility tool for reading and highlighting text.
		\item	Voice Control for Video\cite{voicecontrolvideo} – Media control only.
		\item	VCAT\cite{vcat} – Basic voice commands and limited browsing control.
		\item	Natural Reader Text to Speech\cite{naturalreader} – Multilingual reader for web pages, PDFs, emails.
		
        From the above discussion it is proved that no single extension currently offers browsing control, media control and multilingual accessibility(specially for Bangla) as an unit.
		
	\end{itemize}
	\subsection{Research-based Voice Systems And Academic Works}
	There are several academic and large-scale solutions provide detail insight fro voice-based systems rather than browser extensions.
	
	\textbf{Google Voice Access\cite{googlevoice}} offers accessibility, allow uses to navigate and control the system. While this system is highly effective for mobile devices but it does direct support desktop chrome browsing. Amazon Alexa Browser Integrations\cite{amazonvoice} allow users to do basic search and other tasks but it has dependency on the cloud services and limited to the devices.
	
	\textbf{A Voice Controlled E-commerce Web App\cite{voicecommerce}} focuses on the accessibility of online shopping tasks can be improved through voice commands. It presents that by using voice commands users can perform their shopping from e-commerce site smoothly and efficiently. But this paper does not present any interaction for browser accessibility through voice commands.
	
	Studies on \textbf{Voice-Based HCI\cite{researchstudies}} emphasized different parameters  which are critical for browser-based extension such as  response time, noise handling, and context recognition.
	
	\textbf{Accessibility research\cite{accessibilitystudies}} highlights the necessity of multilingual support, highlighting words during text to speech, and provides integration with different productivity tools.
	
	\textbf{Këpuska \& Bohouta (2017)\cite{kepuska2017comparing}} studied and compared  modern speech recognition APIs (Google, Microsoft, IBM Watson, CMU Sphinx) and referred that cloud-based systems achieve high accuracy while local engines allow offline processing.
	
	\section{Analysis of Findings}
	The analysis of reviewed extensions and academic systems shows that most browser based web extensions offers several features like basic navigation, read selected text, media control, some provides multilingual support but lack of integration with productivity applications. But as an unified system no extension can offer all the features with high response time and support voice commands for Bangla. 
	
	Research studies how accuracy, latency, noise handling, multilingual support with proper design can enhance the productivity of a voice-based solutions which builds the foundation for the proposed system.
	
	\section{Conclusion}
	In summary, the literature review shows that while there exists several extensions which provide partial voice-based browsing control, media control or TTS functionality, no solution offer an unified integrated solution by combining all important features and support for Bangla language. Existing academic research highlights the importance of accuracy, speed, and usability and latency, which are essential to incorporate into the design of the proposed system. The goal of the proposed extension is to bridge these gaps, providing a user-friendly, feature oriented, multilingual, and productivity based voice assistant tool for Chrome web browser.
	







