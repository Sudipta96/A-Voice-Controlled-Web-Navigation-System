


	\chapter{Introduction}
	\section{Overview}
	The rapid growth of the Internet makes the web browser an essential tool for various use cases like accessing information, analysis, performing tasks globally. Traditionally, users use web browser by using keyboards or using mice. This can be time consuming, boring, inconvenience and challenging for people who has limited typing skills or disabilities. Voice based technologies supported by speech recognition and natural language processing bring a ground-breaking change in this area by providing more accessible method of interacting with digital systems.
	
	
	This project focuses on the design and development of the voice controlled Chrome extension which supports Bangla and English. This extension allow users to search online from google and wikipedia, read selected text, can click the highlighted links and many more – all through voice commands. By enabling multiple language, the system enhances the usability to the people in Bangladesh where English is not the the primary language.
	
	
	\section{Limitation of Existing Works}
	Existing voice assistant solutions like Siri or Google Assistant are mostly designed for mobile devices or standalone device with limited integration with desktop browsers. Besides these systems are mostly English centric and lack of usability in Bangla language in some cases. 
	
	
	On the other hand, the existing browser extensions which support voice commands, very few of them integrate features like tab control, form fill up, click highlighted links, read selected text, click important links or inline search. Some of them consume more resources which make them  heavy-weighted browser uses. Very view extensions provide multilingual solutions that support both Banlga and English. So, these limitations motivated me to develop a dedicated browser extension which supports these features with multilingual inclusive.
	
	
	\section{Motivation}
	The motivation for this project grows inside me for the demand of more accessible and efficient browsing methods  for users who performs more multi tasks, need to write long queries, face difficulties to find a particular link in a who page. Similarly, some people has physical difficulties or find interest to listen text content rather than reading whole content. This browser extension with multilingual facilities can bridge this gaps making browsing faster and interesting by providing a lots of features. 
	
	\section{Problem Statements}
	Although the availability of the advanced voice assistant like Siri or
	Google Assistant, there is no lightweight Chrome extension that provides dual-language (English and Bangla) voice control with additional browsing facilities. Users currently lack the ability to:
	\begin{itemize}
		\item	Perform inline search through Google, Wikipedia or YouTube by voice.
		\item	Switch between specific browser tabs using voice recognition.
		\item   Control Media element interaction by voice.
		\item	Click highlighted links to navigate and find the expected links easily.
		\item	Fill up forms, listen the selected text.
	\end{itemize}
	
	This project addresses these issues by proposing and developing a proper voice based Chrome browser extension by including these features with multilingual support in Bangla and English.
	
	
	\section{Objectives}
	The main objectives of this research are:
	\begin{itemize}
		\item To design and develop a voice-based Chrome extension that allows users to interact with the web browser through natural speech commands.
		\item To support multilingual interaction (English and Bengali) so that both local and international users can easily interact with the browser.
		\item To perform different activities through browser like searching content on Google, Wikipedia, form filling, media control(play/pause/mute), click highlighted links etc. using voice commands.
	\end{itemize}
	\section{Research Outline}
	This project is structured into five major chapters:
	\begin{itemize}
		\item	Chapter 1 – Introduction: Provides the background, problem statement, motivation, objectives of the study and research outline.
		\item	Chapter 2 – Literature Review: Discusses about the existing solutions or related works, their features and limitations.
		\item	Chapter 3 – Proposed System Architecture: Explains the architecture, tools, speech recognition techniques and algorithmic approach to achieve the intended functionalities.
		\item	Chapter 4 – Implementation and Results: Describes coding modules, functionalities tested and the experimental outcomes.
		\item	Chapter 5 – Conclusion and Future Works: Summarizes the findings of the project, implications and some important improvements for future.
	\end{itemize}
	
	\section{Conclusion}
	In this chapter, background and motivation of the project are presented along with limitations of the existing solutions. Besides the problem statements and the objectives clearly dictate the necessity of the multilingual voice based Chrome extension. This chapter also provides the research outline which will guide the overall project design and implementation throughout the project. The next chapter will provide the review for the existing solutions and related works to identify the existing gaps which establish the foundation of the proposed system.
	

